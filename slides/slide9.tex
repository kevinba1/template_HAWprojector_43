% Folie 9 - Mathematische Schreibweisen

\begin{frame}{Mathematische Schreibweisen}
In diesem Abschnitt werden paar mathematsche Schreibweisen in LaTeX gezeigt. Außerdem folgt ein Beispiel-Schaltplan.
      
\end{frame}


\subsection{Beispiel 1}
\begin{frame}{Beispiel 1}
Sei $n$ eine natürliche Zahl, $n!=1 \cdot 2 \cdots n$ und
\[
\binom{n}{k} = \frac{n!}{(n-k)!k!} =
\frac{n(n-1)(n-2)\cdots(n-k+1)}{1 \cdots (k-2)(k-1)k} \,.
\]
Für beliebige reelle Zahlen $a$ und $b$ gilt

\begin{equation}
(a+b)^n = \sum_{k=0}^n \binom{n}{k} a^{n-k} b^k \,.
\label{binomischer_Satz}
\end{equation}

Die Gleichung \eqref{binomischer_Satz} wird als {\em Binomischer Satz} bezeichnet.
      
\end{frame}



\subsection{Beispiel 2}
\begin{frame}{Beispiel 2}
Beispiel
Determinante der Vandermonde-Matrix (manuelle Positionierung)\[
\left|\left(\begin{array}{ccccc}
1 & x_1 & x_1^2 & \cdots & x_1^{n-1} \\
\vdots & \vdots & \vdots & \ddots & \vdots \\
1 & x_n & x_n^2 & \cdots & x_n^{n-1}
\end{array}\right)\right| = \prod_{k>j}(x_k-x_j) \,.
\]
      
\end{frame}



\subsection{Beispiel 3}
\begin{frame}{Beispiel 3}
      Ein einfacher Schaltplan mit wenigen Komponenten.
\centering

\begin{circuitikz}
      
      % Spannungsquelle
      \draw (0,0) to [european voltage source, l=$U_1$] (0,0) to (0,2);
      
      % Schalter
      \draw (0,2) to [normal open switch] (2,2);
      
      % Kondensator
      \draw (2,2) to [C, l=$C_1$] (4,2) to (4,1);
      
      % Widerstand 1
      \draw (4,1) to (3,1) to [R, l_=$R_1$] (3,-1) to (4,-1);
      
      % Widerstand 2
      \draw (4,1) to (5,1) to [R, l=$R_2$] (5,-1) to (4,-1);
      
      % Lampe
      \draw (4,-1) to (4,-2) to [lamp] (0,-2) to (0,0);
      
      % Knotenpunkte
      \draw (4,1) node[circ]{};
      \draw (4,-1) node[circ]{};
  \end{circuitikz}

      
\end{frame}