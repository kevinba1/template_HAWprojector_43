% Folie 4 - Tabellen und Abbildungen

\begin{frame}{Tabellen und Abbildungen}
  \begin{itemize}
    \item Benutze \texttt{tabular} für einfache Tabellen --- Gleich folgen drei Beispiele, wie man in LaTeX Tabellen behandeln kann.
    \item Außerdem kann man Abbildungen in verschiedenen Formaten einbinden (JPEG, PNG oder PDF).
    \item Zum einbinden der Abbildungen wird \texttt{includegraphics} benutzt.
  \end{itemize}

\end{frame}



% Folie 5 - Einfache Tabelle
\subsection{Einfache Tabelle}

\begin{frame}{Eine einfache Tabelle}
  Hier folgt eine \alert{einfache} Tabelle:
  \begin{table}
    \centering
    \begin{tabular}{|c|c|}
      \hline
      Größe & Einheit \\
      \hline
      Spannung & Volt (V) \\
      Stromstärke & Ampere (A) \\
      Widerstand & Ohm ($\Omega$) \\
      Leistung & Watt (W) \\
      \hline
    \end{tabular}
    \caption{Eine einfache Tabelle.}
  \end{table}
\end{frame}


% Folie 6 - Komplexe Tabelle
\subsection{Komplexe Tabelle}

\begin{frame}{Eine komplexe Tabelle}
  Hier folgt eine \alert{komplexe} Tabelle:
  \begin{table}
    \centering
    \begin{tabular}{|c|c|c|c|}
      \hline
      \multirow{2}{*}{\textbf{Name}} & \multicolumn{2}{c|}{\textbf{Daten}} & \multirow{2}{*}{\textbf{Gesamt}} \\
      \cline{2-3}
       & \textbf{Wert 1} & \textbf{Wert 2} & \\
      \hline
      Komponente 1 & 10 & 20 & 30 \\
      Komponente 2 & 5 & 15 & 20 \\
      \hline
      \multicolumn{3}{|c|}{\textbf{Summe}} & 50 \\
      \hline
    \end{tabular}
    \caption{Eine komplexe Tabelle.}
  \end{table}
\end{frame}

% Folie 8 - Abbildungen
\subsection{Ausgelagerte Tabelle}

\begin{frame}{Eine augelagerte Tabelle}
  \begin{itemize}[label=\textbullet]
    \item In \LaTeX kann man Tabellen auch in eigene \texttt{.tex-Datein} auslagern.
    \item Das spart Platz und ist bei der Bearbeitung übersichtlicher.
  \end{itemize}
  
  \begin{table}
  \centering
    
  \begin{tabular}{|c|c|c|}
      \hline
      \rowcolor[HTML]{EFEFEF}
      \textbf{Komponente} & \textbf{Parallel} & \textbf{Reihe} \\
      \hline
      Widerstand & $\frac{1}{R_{\text{ges}}} = \sum \frac{1}{R_i}$ & $R_{\text{ges}} = \sum R_i$ \\
      \hline
      \rowcolor[HTML]{EFEFEF}
      Kondensator & $C_{\text{ges}} = \sum C_i$ & $\frac{1}{C_{\text{ges}}} = \sum \frac{1}{C_i}$ \\
      \hline
      Spule & $\frac{1}{L_{\text{ges}}} = \sum \frac{1}{L_i}$ & $L_{\text{ges}} = \sum L_i$ \\
      \hline
    \end{tabular}
  \caption{Eine ausgelagerte Tabelle.}
  \end{table}
  
\end{frame}


% Folie 8 - Abbildungen
\subsection{Abbildungen}

\begin{frame}{Abbildungen}
  So kann eine Abbildung hinzugefügt werden:
  \begin{figure}
  \includegraphics[width=8cm]{images/latex_logo.png}
  \caption{Untertitel.}
  \end{figure}
\end{frame}
