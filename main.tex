% !TEX root = main.tex
%
% An example of a projector presentation based on the Corporate Design 2022
% of the Hamburg University of Applied Sciences.
%
% - Author:  Kevin Balybin
% - License: CC BY-SA 4.0 (https://creativecommons.org/licenses/by-sa/4.0/)
% 
% - Based on the repository https://github.com/jbeyerstedt/LaTeX-HAW-Beamertheme
% - Contains sections of https://github.com/matze/mtheme
%
%

\documentclass[aspectratio=43,10pt]{beamer}

\usetheme{HAWprojector}
\usepackage{styles/HAWprojectorstyle}



\addbibresource{literature.bib}

% CUSTOM SETTINGS
% \setcounter{tocdepth}{3}
% \setcounter{secnumdepth}{3}
% \usepackage{tabularx}
\renewcommand{\arraystretch}{1.5}   % less crammed tables

% MANUAL HYPHENATION (whitespace separated items)
\hyphenation{}

% CUSTOM UNITS
\sisetup{locale = DE}
% \DeclareSIUnit \fps {fps}   % a custom unit (usage: \fps{25})

% use a) ABBREVIATIONS
% \nomenclature[UDP]{UDP}{User Datagram Protocol}
\makenomenclature

% or b) GLOSSARY AND ACRONYMS
% \newglossaryentry{CBC}{name=CBC, description={Cipher Block Chaining. Betriebsart des symmetrischen Kryptografie-Algorithmus AES}}
% \newacronym{iot}{IoT}{Internet of Things}
% \newacronym[see={[Glossary:]{ota_long}}]{ota}{OTA}{Over the Air\glsadd{ota_long}}
\makeglossaries




%-----------------------%
% DOCUMENT / TITLE PAGE %
%-----------------------%

\title[HAW Hamburg Projector]{Presentation template}
\subtitle{Template for presentations}
\author{Arthur Dent}
\institute{Hochschule für Angewandte Wissenschaften Hamburg}
\date{25.05.2023}

\faculty{Abteilung oder Fakultät}
\footerinfo{Kleine Fußnote (mehr Infos\dots)}


%-----------------------%
% TITLE PAGE            %
%-----------------------%

\begin{document}
\maketitle


%-----------------------%
% OUTLINE / TOC         %
%-----------------------%

\begin{frame}{Gliederung}
  \tableofcontents
%  \tableofcontents[hideallsubsections]
\end{frame}

%-----------------------%
% SLIDES                %
%-----------------------%

% Folie 1 - Disclaimer
\section{Disclaimer}
% Folie 1 - Disclaimer

\begin{frame}{Disclaimer}
  Diese Präsentation dient als \LaTeX-Präsentationsvorlage im 4:3-Format basierend auf dem Corporate Design der Hochschule für Angewandte Wissenschaften Hamburg aus dem Jahr 2022. Die enthaltenen Beispiele dienen lediglich zur Veranschaulichung von \LaTeX-Funktionen. Bitte beachten Sie, dass \LaTeX-Kenntnisse erforderlich sind, um die Vorlage effektiv zu nutzen. Für Fehler oder Schäden, die durch die Verwendung der Vorlage oder der Beispiele entstehen könnten, wird keine Haftung übernommen.
\end{frame}


% Folie 2 - Beispiele für Texte
\section{Beispiele für Texte}
% Folie 2 - Beispiele für Texte

\begin{frame}{Beispiele für Texte}
Hier steht Lauftext. Begriffe können \emph{kursiv hervorgehoben} werden. Bei Präsentationen mit einem Projektor wird diese \alert{deutlichere Hervorhebung} empfohlen.
\newline
\newline
\newline
So könnte eine Aufzählung aussehen:
\begin{itemize}[label=\textbullet]
  \item Mit \texttt{itemize} können solche Aufzählungen erstellt werden.
  \item Mit \texttt{item} können weitere Stichpunkte hinzugefügt werden.
\end{itemize}
\end{frame}


% Folie 3 - Textblöcke
\section{Textblöcke}
% Folie 3 - Textblöcke

\begin{frame}{Textblöcke}

\begin{block}{Einfacher Block}
  Hier steht der Inhalt.
\end{block}

\begin{alertblock}{Alert-Block}
  Hier steht der Inhalt.
\end{alertblock}

\begin{exampleblock}{Example-Block}
  Hier steht der Inhalt.
\end{exampleblock}

\begin{example}
  \begin{itemize}
    \item Auflistung innerhalb eines Text-Blocks.
    \item \dots
  \end{itemize}
\end{example}

\end{frame}

% Folie 4 - Tabellen und Abbildungen (in einer TeX)
\section{Tabellen und Abbildungen}
% Folie 4 - Tabellen und Abbildungen

\begin{frame}{Tabellen und Abbildungen}
  \begin{itemize}
    \item Benutze \texttt{tabular} für einfache Tabellen --- Gleich folgen drei Beispiele, wie man in LaTeX Tabellen behandeln kann.
    \item Außerdem kann man Abbildungen in verschiedenen Formaten einbinden (JPEG, PNG oder PDF).
    \item Zum einbinden der Abbildungen wird \texttt{includegraphics} benutzt.
  \end{itemize}

\end{frame}



% Folie 5 - Einfache Tabelle
\subsection{Einfache Tabelle}

\begin{frame}{Eine einfache Tabelle}
  Hier folgt eine \alert{einfache} Tabelle:
  \begin{table}
    \centering
    \begin{tabular}{|c|c|}
      \hline
      Größe & Einheit \\
      \hline
      Spannung & Volt (V) \\
      Stromstärke & Ampere (A) \\
      Widerstand & Ohm ($\Omega$) \\
      Leistung & Watt (W) \\
      \hline
    \end{tabular}
    \caption{Eine einfache Tabelle.}
  \end{table}
\end{frame}


% Folie 6 - Komplexe Tabelle
\subsection{Komplexe Tabelle}

\begin{frame}{Eine komplexe Tabelle}
  Hier folgt eine \alert{komplexe} Tabelle:
  \begin{table}
    \centering
    \begin{tabular}{|c|c|c|c|}
      \hline
      \multirow{2}{*}{\textbf{Name}} & \multicolumn{2}{c|}{\textbf{Daten}} & \multirow{2}{*}{\textbf{Gesamt}} \\
      \cline{2-3}
       & \textbf{Wert 1} & \textbf{Wert 2} & \\
      \hline
      Komponente 1 & 10 & 20 & 30 \\
      Komponente 2 & 5 & 15 & 20 \\
      \hline
      \multicolumn{3}{|c|}{\textbf{Summe}} & 50 \\
      \hline
    \end{tabular}
    \caption{Eine komplexe Tabelle.}
  \end{table}
\end{frame}

% Folie 8 - Abbildungen
\subsection{Ausgelagerte Tabelle}

\begin{frame}{Eine augelagerte Tabelle}
  \begin{itemize}[label=\textbullet]
    \item In \LaTeX kann man Tabellen auch in eigene \texttt{.tex-Datein} auslagern.
    \item Das spart Platz und ist bei der Bearbeitung übersichtlicher.
  \end{itemize}
  
  \begin{table}
  \centering
    
  \begin{tabular}{|c|c|c|}
      \hline
      \rowcolor[HTML]{EFEFEF}
      \textbf{Komponente} & \textbf{Parallel} & \textbf{Reihe} \\
      \hline
      Widerstand & $\frac{1}{R_{\text{ges}}} = \sum \frac{1}{R_i}$ & $R_{\text{ges}} = \sum R_i$ \\
      \hline
      \rowcolor[HTML]{EFEFEF}
      Kondensator & $C_{\text{ges}} = \sum C_i$ & $\frac{1}{C_{\text{ges}}} = \sum \frac{1}{C_i}$ \\
      \hline
      Spule & $\frac{1}{L_{\text{ges}}} = \sum \frac{1}{L_i}$ & $L_{\text{ges}} = \sum L_i$ \\
      \hline
    \end{tabular}
  \caption{Eine ausgelagerte Tabelle.}
  \end{table}
  
\end{frame}


% Folie 8 - Abbildungen
\subsection{Abbildungen}

\begin{frame}{Abbildungen}
  So kann eine Abbildung hinzugefügt werden:
  \begin{figure}
  \includegraphics[width=8cm]{images/latex_logo.png}
  \caption{Untertitel.}
  \end{figure}
\end{frame}



% Folie 9 - Mathematsche Schreibweisen
\section{Mathematsche Schreibweisen}
% Folie 9 - Mathematische Schreibweisen

\begin{frame}{Mathematische Schreibweisen}
In diesem Abschnitt werden paar mathematsche Schreibweisen in LaTeX gezeigt. Außerdem folgt ein Beispiel-Schaltplan.
      
\end{frame}


\subsection{Beispiel 1}
\begin{frame}{Beispiel 1}
Sei $n$ eine natürliche Zahl, $n!=1 \cdot 2 \cdots n$ und
\[
\binom{n}{k} = \frac{n!}{(n-k)!k!} =
\frac{n(n-1)(n-2)\cdots(n-k+1)}{1 \cdots (k-2)(k-1)k} \,.
\]
Für beliebige reelle Zahlen $a$ und $b$ gilt

\begin{equation}
(a+b)^n = \sum_{k=0}^n \binom{n}{k} a^{n-k} b^k \,.
\label{binomischer_Satz}
\end{equation}

Die Gleichung \eqref{binomischer_Satz} wird als {\em Binomischer Satz} bezeichnet.
      
\end{frame}



\subsection{Beispiel 2}
\begin{frame}{Beispiel 2}
Beispiel
Determinante der Vandermonde-Matrix (manuelle Positionierung)\[
\left|\left(\begin{array}{ccccc}
1 & x_1 & x_1^2 & \cdots & x_1^{n-1} \\
\vdots & \vdots & \vdots & \ddots & \vdots \\
1 & x_n & x_n^2 & \cdots & x_n^{n-1}
\end{array}\right)\right| = \prod_{k>j}(x_k-x_j) \,.
\]
      
\end{frame}



\subsection{Beispiel 3}
\begin{frame}{Beispiel 3}
      Ein einfacher Schaltplan mit wenigen Komponenten.
\centering

\begin{circuitikz}
      
      % Spannungsquelle
      \draw (0,0) to [european voltage source, l=$U_1$] (0,0) to (0,2);
      
      % Schalter
      \draw (0,2) to [normal open switch] (2,2);
      
      % Kondensator
      \draw (2,2) to [C, l=$C_1$] (4,2) to (4,1);
      
      % Widerstand 1
      \draw (4,1) to (3,1) to [R, l_=$R_1$] (3,-1) to (4,-1);
      
      % Widerstand 2
      \draw (4,1) to (5,1) to [R, l=$R_2$] (5,-1) to (4,-1);
      
      % Lampe
      \draw (4,-1) to (4,-2) to [lamp] (0,-2) to (0,0);
      
      % Knotenpunkte
      \draw (4,1) node[circ]{};
      \draw (4,-1) node[circ]{};
  \end{circuitikz}

      
\end{frame}


% Folie 10 - References
\section{References}
\begin{frame}{References}
    So können Verweise dargestellt werden. Wichtig ist dabei, dass die Datei \alert{\emph{literature.bib}} den richtigen Inhalt hat. \cite{ConcreteMath,Simpson,Er01,greenwade93}
  \end{frame}


% Folie 11 - References i
\begin{frame}[allowframebreaks]{References}
  \printbibliography
\end{frame}


\end{document}
